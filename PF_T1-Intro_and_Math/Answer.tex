\documentclass[12pt, a4paper, oneside]{article}
\usepackage{amsmath, amsthm, amssymb, bm, graphicx, hyperref, mathrsfs, xcolor, mdframed, mathrsfs}
\newcommand{\bhref}[2]{
    \href{#1}{\color{blue}{#2}}
}
\newcommand{\vecf}{\mathbf{f}}

\linespread{1.5}
\newcounter{problemname}
\newenvironment{problem}{\begin{shadecolorbox}
    \stepcounter{problemname}\par\noindent\textsc{Q\arabic{problemname}. }}
    {\end{shadecolorbox}\par}
\newenvironment{note}{\begin{notecolorbox}
    \par\noindent\textsc{Note of Problem \arabic{problemname}. }}
    {\end{notecolorbox}\par}
\newenvironment{warn}{\begin{warncolorbox}
    \par\noindent\textsc{Warning!}}
    {\end{warncolorbox}\par}
\newenvironment{solution}{\par\noindent\emph{Solution. }}{\par}

\definecolor{shadecolor}{RGB}{241, 241, 255}
\definecolor{notecolor}{RGB}{230, 242, 252}
\definecolor{warncolor}{RGB}{255, 211, 197}

\newmdenv[linecolor=white,backgroundcolor=notecolor]{notecolorbox}
\newmdenv[linecolor=white,backgroundcolor=warncolor]{warncolorbox}
\newmdenv[linecolor=white,backgroundcolor=shadecolor]{shadecolorbox}


\begin{document}

\title{Phase Field Tutorial 1: Exercise Answer}
\author{AMoment}
\date{\today}
\maketitle

\begin{problem}
Suppose there are three functions:\(\phi,\psi : \mathbb{R}^3 \to \mathbb{R}\) and \(\vecf : \mathbb{R}^3 \to \mathbb{R}^3\).
Please prove the following indentities:
\[\nabla\left( \phi\psi \right) = \phi\nabla\psi + \psi\nabla\phi\]
\[\nabla\cdot\left( \phi\vecf \right) = \nabla\phi\cdot\vecf + \phi\nabla\cdot\vecf.\]
\end{problem}
\medskip
\begin{proof}
    By the definition of the \(\nabla\):
    \[
        \nabla = \begin{bmatrix}
            \frac{\partial }{\partial x_1} \\
            \frac{\partial }{\partial x_2} \\
            \frac{\partial }{\partial x_3}
        \end{bmatrix} = \left[
            \frac{\partial }{\partial x_1},
            \frac{\partial }{\partial x_2},
            \frac{\partial }{\partial x_3}
            \right]^\mathsf{T},
    \]
    thus we have:
    \begin{align*}
        \nabla\left( \phi\psi \right)
         & = \left[
            \frac{\partial \left( \phi\psi \right)}{\partial x_1} ,\,
            \frac{\partial \left( \phi\psi \right)}{\partial x_2} ,\,
            \frac{\partial \left( \phi\psi \right)}{\partial x_3}
        \right]^\mathsf{T}                           \\
         & =\left[
            \frac{\partial \phi}{\partial x_1}\psi + \phi \frac{\partial \psi}{\partial x_1} ,\,
            \frac{\partial \phi}{\partial x_2}\psi + \phi \frac{\partial \psi}{\partial x_2} ,\,
            \frac{\partial \phi}{\partial x_3}\psi + \phi \frac{\partial \psi}{\partial x_3}
        \right]^\mathsf{T}                           \\
         & =\left[
            \psi\frac{\partial \phi}{\partial x_1} ,\,
            \psi\frac{\partial \phi}{\partial x_2} ,\,
            \psi\frac{\partial \phi}{\partial x_3}
            \right]^\mathsf{T} +
        \left[
            \phi \frac{\partial \psi}{\partial x_1} ,\,
            \phi \frac{\partial \psi}{\partial x_2} ,\,
            \phi \frac{\partial \psi}{\partial x_3}
        \right]^\mathsf{T}                           \\
         & = \psi\left[
            \frac{\partial \phi}{\partial x_1} ,\,
            \frac{\partial \phi}{\partial x_2} ,\,
            \frac{\partial \phi}{\partial x_3}
            \right]^\mathsf{T} +
        \phi\left[
            \frac{\partial \psi}{\partial x_1} ,\,
            \frac{\partial \psi}{\partial x_2} ,\,
            \frac{\partial \psi}{\partial x_3}
        \right]^\mathsf{T}                           \\
         & =        \phi\nabla\psi + \psi\nabla\phi,
    \end{align*}
    and:
    \begin{align*}
        \nabla\cdot\left( \phi\vecf \right)
         & =
        \left[
            \frac{\partial }{\partial x_1} ,\,
            \frac{\partial }{\partial x_2} ,\,
            \frac{\partial }{\partial x_3}
            \right]^\mathsf{T}
        \cdot
        \left[
            \phi{f_1} ,\,
            \phi{f_2} ,\,
            \phi{f_3}
        \right]^\mathsf{T}                                                              \\
         & =
        \frac{\partial }{\partial x_1}\phi{f_1} +
        \frac{\partial }{\partial x_2}\phi{f_2} +
        \frac{\partial }{\partial x_3}\phi{f_3}                                         \\
         & =
        \frac{\partial \phi}{\partial x_1}f_1 + \frac{\partial {f_1}}{\partial x_1}\phi+
        \frac{\partial \phi}{\partial x_2}f_2 + \frac{\partial {f_2}}{\partial x_2}\phi+
        \frac{\partial \phi}{\partial x_3}f_3 + \frac{\partial {f_3}}{\partial x_3}\phi \\
         & =
        \left(
        \frac{\partial \phi}{\partial x_1}f_1 +
        \frac{\partial \phi}{\partial x_2}f_2 +
        \frac{\partial \phi}{\partial x_3}f_3 \right)
        +
        \left(
        \phi\frac{\partial {f_1}}{\partial x_1}+
        \phi\frac{\partial {f_2}}{\partial x_2}+
        \phi\frac{\partial {f_3}}{\partial x_3}  \right)                                \\
         & = \nabla\phi\cdot\vecf + \phi\nabla\cdot\vecf,
    \end{align*}
\end{proof}
\begin{note}
    Here are some detail about this question and the answer:

    \begin{enumerate}
        \item To understand the behavior of nabla operator, although this operator is actually operating on functions, we can still consider this operation as some vector multiplication. To see what will interact with nabla's each components, we can check the function's codomain.

              If the codomain of the function is a scalar space (that is, \(\mathbb{R}\)), then we know that we can only take gradient of this function, like `scalar multiply vector' to obtain a `vector' (field);

              If the codomain of the function is a vector space (that is, \(\mathbb{R}^3\)), then we can take  divergence or curl of this field, as we can do vector's dot product and cross product, respectively.

        \item Sometimes you can see the authors use
              \[
                  \nabla =
                  \frac{\partial }{\partial x_1}\hat{x_1} +
                  \frac{\partial }{\partial x_2}\hat{x_2}+
                  \frac{\partial }{\partial x_3}\hat{x_3}
              \]
              instead of directly use the vector form. That's totally okay to do so, just to notice that the base vectors (\(\hat{x_1}\), \(\hat{x_2}\) and \(\hat{x_3}\)) are used `globally', and they are (to some extent, can be viewed as) the base vectors of the domain of the functions you are operating on.

        \item Please notice that the notation of the base vectors are not really restricted, as long as you are consistent with the context. For example, if you prefer \(x\), \(y\) and \(z\) instead of \(\hat{x_1}\), \(\hat{x_2}\) and \(\hat{x_3}\), it's totally okay as long as you use them consistently.
    \end{enumerate}
\end{note}

\begin{problem}
Please try to subsitute the following free energy functional into Cahn-Hilliard equation get the gonvering equation:
\[F\left( x,c,\nabla c \right) = \int_{\Omega} f(c) + \frac{\kappa}{2}\left( \nabla c \right)^2 \,\mathrm{d}v.\]
Notice that the mobility \(M_{ij}\) of Cahn-Hilliard equation is considered as a constant:
\begin{align*}
    \frac{\partial c_i}{\partial t}
     & =   \nabla \cdot M_{ij} \nabla \frac{\delta F}{\delta c_j \left( r,t \right)} \\
     & = M\nabla^2 \frac{\delta F}{\delta c_j \left( r,t \right)}.
\end{align*}
\end{problem}

\begin{solution}
    The variational derivative of a functional in the following form:
    \[
        J[y] = \int_{\Omega} L(x,y(x),\nabla y(x))\,\mathrm{d}\omega
    \]
    is:
    \[
        \frac{\delta J[y]}{\delta y} = \frac{\partial L}{\partial y}-
        \nabla\cdot\frac{\partial L}{\partial \nabla{y}}.
    \]
    Thus, by taking variational derivative of the free energy functional: \(F[c] = F(x,c,\nabla c)\),
    and define free energy density \(f_t(x,c,\nabla c) = f(c) + \frac{\kappa}{2}\left( \nabla c \right)^2\),
    we will get:
    \begin{align*}
        \frac{\delta F}{\delta c_j}
         & = \frac{\partial f_t}{\partial c_j}-
        \nabla\cdot\frac{\partial f_t}{\partial \nabla{c_j}}.                               \\
         & = \frac{\partial f}{\partial c_j} - \nabla\cdot \left( \kappa \nabla c_j \right) \\
         & = \frac{\partial f}{\partial c_j} - \kappa \nabla^2 c_j.
    \end{align*}
    So, the final gonvering equation will be:
    \[\frac{\partial c_i}{\partial t} = M \nabla^2 \left( \frac{\partial f}{\partial c_j} - \kappa \nabla^2 c_j \right).\]

\end{solution}

\begin{note}
    \begin{enumerate}
        \item The variational derivative of a functional \(F\) with gradient of function (\(\nabla{c}\))  as its derivative to the variable ( \(y'\) ) should
              replace \( \frac{\mathrm{d} }{\mathrm{d} x}()\) with \(\nabla \cdot()\).
              You can refer to the \bhref{https://en.wikipedia.org/wiki/Functional_derivative\#Formula}{Wiki} for more information.
        \item Take a look at the integrand of the free energy functional, you will find that \(\left( \nabla c \right)^2\) should be a scalar. Indeed this is the inner
              product of the gradient of concentration. Fortunately, the derivative of \(\nabla c \cdot \nabla c \) with respect to the vector \(\nabla c\) is still \(2\nabla c\).
    \end{enumerate}

\end{note}


\subsubsection*{Derivation of E-L equation}
For more details, for example, how can we derive the E-L equation of the functional
\[
    J[c] = \int_{\Omega} L(x,c(x),\nabla c(x))\,\mathrm{d}\omega,
\]
we can mimic the E-L derivation we discussed before.

First, as we did before, we consider that there should be a function, called \(\bar{c}\), satisfies the variational condition, that is,
\[
    \frac{\delta J[c]}{\delta c}\vert_{\bar{c}} =0.
\]
Then we can always find a set of function that take \(0\) value on the boundary of the region \(\Omega\): \(\partial \Omega\). We \emph{arbitrarily} take one such function denoted as \(\eta\), then every variation on the function \(c\) could be represented as:
\[
    \delta c = \varepsilon\eta,
\]
where \(\varepsilon\) is a infinitesimal real number that we will let it to approach \(0\) to force \(\delta c\) to approach \(0\).

Then we argue that, as we have already supposed that \(\bar{c}\) satisfies the variational condition, then any non-zero variation applies to this function will break the condition. Now we rewrite the functional. The functional is basically `a function of function', now that we've selected a function \(\bar{c}\) and we use variation on this function, the functional can be rewrite as:
\[
    \Phi(\varepsilon) = J[\bar{c} + \varepsilon \eta] = \int_\Omega f(\bar{c} + \varepsilon \eta) + \frac{\kappa}{2}(\nabla (\bar{c} + \varepsilon \eta))^2\,\mathrm{d}\omega ,
\]
then we can see that, if \(\varepsilon = 0\), then the functional will be  \(J[\bar{c}]\) and hence satisfy the condition. To describe the condition, we can take derivative on \(\Phi(\varepsilon)\):
\[
    \begin{aligned}
        \Phi'(\varepsilon)
         & = \int_\Omega \frac{\mathrm{d}f(\bar{c} + \varepsilon \eta)}{\mathrm{d}\varepsilon} + \frac{\kappa}{2}\frac{\mathrm{d}(\nabla (\bar{c} + \varepsilon \eta))^2}{\mathrm{d}\varepsilon}\,\mathrm{d}\omega                              \\
         & =\int_\Omega f' \eta + \frac{\kappa}{2} \frac{\mathrm{d}}{\mathrm{d}\varepsilon}(\nabla\bar{c}\cdot\nabla\bar{c} + \nabla(\varepsilon\eta)\cdot\nabla(\varepsilon\eta) + 2\varepsilon\nabla\bar{c}\cdot\nabla\eta)\,\mathrm{d}\omega \\
         & =\int_\Omega f' \eta + \kappa (\varepsilon\nabla\eta\cdot\nabla\eta + \nabla\bar{c} \cdot \nabla\eta)\,\mathrm{d}\omega,
    \end{aligned}
\]
The second step is from expansion of dot products, and as \(\varepsilon\) is a scalar variable, it can be take out from the gradients; The third step is because the \(\bar{c}\) is constant vector fields and hence the dot product is constant scalar field, hence will disappear after derivative.

Now we let \(\varepsilon = 0\), because the definition of \(\eta\) and \(\bar{c}\), we get:

\[
    \Phi'(0) = \int_\Omega f' \eta + \kappa (\nabla\bar{c} \cdot \nabla\eta)\,\mathrm{d}\omega = 0,
\]

What we want to show is that there is something independent of \(\eta\) and must be zero to let the integral to be zero. Now we must deal with the dot product of two gradients. We shall apply the equality:

\[\nabla\cdot(f\mathbf{v}) = f\nabla\cdot\mathbf{v} + \nabla f\cdot \mathbf{v},\]

notice that \(\eta\) is a scalar field and \(\nabla\bar{c}\) is a vector field, by rearrange the equation above, we can substitute the corresponding part and obtain:

\[
    \begin{aligned}
        \Phi'(0)
         & = \int_\Omega f' \eta + \kappa (\nabla\bar{c} \cdot \nabla\eta)\,\mathrm{d}\omega                                                                   \\
         & = \int_\Omega f' \eta + \kappa (\nabla\cdot(\eta\nabla\bar{c}) - \eta(\nabla\cdot \nabla\bar{c}))\,\mathrm{d}\omega                                 \\
         & = \int_\Omega f' \eta -\kappa \eta(\nabla\cdot \nabla\bar{c})\,\mathrm{d}\omega+ \kappa \int_\Omega\nabla\cdot(\eta\nabla\bar{c})\,\mathrm{d}\omega \\
    \end{aligned}
\]

Now we focus on the latter integral. We shall now apply the divergence theorem:

\[
    \int_\Omega\nabla\cdot(\eta\nabla\bar{c})\,\mathrm{d}\omega = \oint_{\partial\Omega}\eta\nabla\bar{c}\cdot \hat{n}\,\mathrm{d}S.
\]

Notice that, on the boundary \(\partial\Omega\), \(\eta\) is zero. Hence this integral should be zero, and the original derivative will be:

\[
    \begin{aligned}
        \Phi'(0) & = \int_\Omega f' \eta -\kappa \eta(\nabla\cdot \nabla\bar{c})\,\mathrm{d}\omega \\
                 & = \int_\Omega( f' -\kappa (\nabla\cdot \nabla\bar{c}))\eta\,\mathrm{d}\omega    \\
                 & = 0
    \end{aligned}
\]

Then, as \(\eta\) is arbitrary function satisfying the condition, to let the derivative be zero, the inner part must be zero, and we get the functional derivative of the functional:

\[\begin{aligned}
        \frac{ \delta J[c]}{\delta c} & =  f' -\kappa (\nabla\cdot \nabla c)               \\
                                      & = \frac{\partial f}{\partial c} - \kappa \Delta c.
    \end{aligned}\]

\subsubsection*{Another way to calculate the E-L equation}
In \bhref{https://blog.csdn.net/kuailezhizi1996/article/details/129011165}{this blog}, there is another method to derive the same form we would obtain from the variational derivative:
\begin{align*}
    \delta F & = \delta \int_{\Omega} \left(  f(c) + \frac{\kappa}{2}\left( \nabla c \right)^2\right) \,\mathrm{d}\omega                                                                                               \\
             & = \delta \int_{\Omega} \left(  f(c) + \frac{\kappa}{2}\left|\nabla c \right|^2 \right) \,\mathrm{d}\omega                                                                                               \\
             & = \int_{\Omega} \left( \delta f(c) + \kappa\,\delta\partial_i c\, \partial_i c \right) \,\mathrm{d}\omega                                                                                               \\
             & = \int_{\Omega} \left(  \frac{\partial f(c)}{\partial c} \delta c + \kappa\,\partial_i\delta c\, \partial_i c \right) \,\mathrm{d}\omega                                                                \\
             & = \int_{\Omega} \left( \frac{\partial f(c)}{\partial c} \delta c + \kappa\left[ \partial_i (\partial_i c \delta c) - \partial_i \partial_i c \delta c \right] \right) \,\mathrm{d}\omega                \\
             & = \int_{\Omega} \left( \frac{\partial f(c)}{\partial c} - \kappa \partial_i \partial_i c \right)\,\delta c  \mathrm{d}\omega + \int_{\Omega}\kappa \partial_i (\partial_i c \delta c)\,\mathrm{d}\omega \\
             & = \int_{\Omega} \left( \frac{\partial f(c)}{\partial c} - \kappa \partial_{ii} c  \right)\,\delta c \mathrm{d}\omega + \int_{\partial\Omega}\kappa \partial_i c n_i \,\delta c\mathrm{d}\omega          \\
             & = \int_{\Omega} \left( \frac{\partial f(c)}{\partial c} - \kappa \nabla^2 c  \right) \,\delta c\mathrm{d}\omega + \int_{\partial\Omega}\kappa \nabla c \cdot \mathbf{n} \,\delta c\mathrm{d}\omega
\end{align*}
Now as the gradient is perpendicular to the normal vector along the edge of the solving domain \(\partial\Omega\), the last term \(\nabla c \cdot \mathbf{n}\) should be 0.
We now again obtain the variational derivative of free energy functional.

\end{document}